\chapter*{Sommario} % senza numerazione
\label{sommario}

\addcontentsline{toc}{chapter}{Sommario} % da aggiungere comunque all'indice

Questa tesi di laurea tratta dell'esperienza di tirocinio svolta nell'arco di tempo intercorso tra febbraio e giugno del 2019 presso l'azienda Digicando.

La contraffazione causa ogni anno la perdita di svariati miliardi di euro in diversi settori produttivi e di posti di lavoro. La produzione di merci contraffatte è un fenomeno in aumento. La vendita di merci contraffatte, oltre a comportare una perdita economica per i produttori autentici, comporta un danno d'immagine per gli stessi, data la scarsa qualità dei prodotti contraffatti ma ritenuti originali dagli acquirenti. Quando si acquista un prodotto non è infatti generalmente possibile verificarne l'autenticità e la qualità. Le recensioni giocano un ruolo fondamentale nella scelta di un prodotto, ma non è possibile verificarne la genuinità. Per i produttori è inoltre difficile vigilare sulla catena logistica ed sul lavoro dei venditori.

L'azienda Digicando offre un servizio che mira a risolvere i problemi sopraelencati. Attualmente il sistema è sviluppato con un'architettura client-server centralizzata, ma l'azienda ha avviato lo sviluppo del suo servizio basato sulle tecnologie della blockchain Ethereum. L'obiettivo di questa transizione è fornire un servizio trasparente che mantenga una cronologia immodificabile dei trasferimenti di un prodotto nella catena di distribuzione, aperto al controllo di chiunque.

L'implementazione su blockchain del servizio di Digicando, pur essendo in uno stato avanzato dello sviluppo, risulta limitato da alcune mancanze, prima fra tutte l'impossibilità di gestire le azioni riservate a Digicando o ad un produttore da più indirizzi Ethereum ed i permessi ad essi associati. Si è quindi reso necessario lo sviluppo di un sistema di gestione dei ruoli e dei permessi orientato agli indirizzi.

Il sistema di gestione dei ruoli è stato sviluppato nel linguaggio di programmazione Solidity, proprio della blockchain Ethereum, e comprende due contratti. Il primo contratto funge da registro ed è destinato alla semplice memorizzazione dei ruoli associati a dei determinati indirizzi. Il secondo contratto fornisce una funzione che permette di verificare la validità di strutture gerarchiche complesse.

Il sistema sviluppato soddisfa i requisiti definiti dall'azienda ed una volta integrato con l'implementazione presente ha permesso di risolvere il problema descritto precedentemente. I contratti di Digicando e dei produttori sono ora gestibili da più indirizzi ed è possibile rimuoverne o aggiungerne altri.