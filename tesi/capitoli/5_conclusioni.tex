\chapter{Conclusioni}
\label{cha:conclusioni}
Gli obiettivi principali del tirocinio erano comprendere i principi alla base di una blockchain ed i relativi strumenti di programmazione, sviluppare capacità di analisi, di progettazione software e di lavorare in team. Terminato il tirocinio gli obiettivi possono essere considerati raggiunti. In particolare, sono state acquisite le competenze per lo sviluppo di applicazioni basate sulla blockchain.

I requisiti individuati dall'azienda per il sistema sono stati soddisfatti ed il sistema sviluppato si è integrato correttamente con le parti del servizio già implementate risolvendo il problema dei ruoli, descritto nel capitolo \ref{cha:problema-ruoli}. La soluzione sviluppata permette di gestire i contratti da molteplici indirizzi, senza la necessità di avere un indirizzo condiviso da tutti gli utenti autorizzati. Questo porta ad una maggiore sicurezza del sistema, permettendo in caso di necessità di revocare l'autorizzazione a determinati utenti.
Il contratto \emph{Management} presenta una criticità nota, dovuta al fatto che durante lo sviluppo si è data priorità alla semplicità, piuttosto che alle funzionalità. Questa criticità permette ad un utente autorizzato come manager che diventi malevolo di rimuovere tutti gli altri manager, ottenendo il controllo esclusivo del contratto.

L'implementazione blockchain del servizio di Digicando verrà resa open-source in futuro, quindi chiunque avesse la necessità di un sistema di gestione dei ruoli su blockchain Ethereum potrà adottare il sistema descritto in questa tesi ed integrarlo nel proprio progetto.